\chapter{安装 Visual Studio Code}
\label{Installation_VSCode}

由于 FlatPak (或 Snap) 的限制,如 Visual Studio Code (VSCode, VSC) 和 JetBrains 系列产品等代码编辑器或 IDE 最好直接安装在 Linux 系统中。以下安装流程也将按照这一方式安装来提供指导。

\section{在 Windows 下安装 Visual Studio Code}
\label{Installation_VSCode_Windows}

在 Windows 下安装 VSCode 有两种不同的方式:通过 \textbf{Microsoft Store} 和 通过 \textbf{安装包}。

\subsection{通过 Microsoft Store 安装 Visual Studio Code}
\label{Installation_VSCode_Windows_Store}

如果你使用自己的微软账户登录了你的 Windows,那么使用 \textbf{Microsoft Store} 安装 Visual Studio Code 可能是最简单便捷的方法:
\begin{enumerate}
    \item 打开 Microsoft Store;
    \item 在搜索区域搜索 Visual Studio Code;
    \item 选择蓝色图标的 Visual Studio Code,或者如果你喜欢尝鲜,选择绿色图标的测试版;
    \item 点击安装,等待安装完成即可
\end{enumerate}

\subsection{通过安装包安装 Visual Studio Code}
\label{Installation_VSCode_Windows_Installer}

另一种比较传统的方法是使用安装程序来安装,这允许你对安装过程进行一些自定义。
\begin{enumerate}
    \item \href{https://code.visualstudio.com/download}{访问 Visual Studio Code 的下载页面};
    \item 选择对应的安装包,或者如果你想进一步定制,在下方小字部分选择 ``User Installer'' 或 ``System Installer'',\textbf{注意选择和你的机器处理器架构对应的安装包!}
    \item 运行下载的安装包,根据指引逐步完成安装即可
\end{enumerate}


\section{在 Linux (RHEL 系发行版) 下安装 Visual Studio Code}
\label{Installation_VSCode_RHEL}

在 RHEL 系发行版下,可以通过如下方式将 VSCode 加入软件源,从而便于安装和更新。

首先,将 VSCode 存储库和密钥安装到系统:
\begin{lstlisting}[language=bash]
    sudo rpm --import https://packages.microsoft.com/keys/microsoft.asc &&
    echo -e "[code]\nname=Visual Studio Code\nbaseurl=https://packages.microsoft.com/yumrepos/vscode\nenabled=1\nautorefresh=1\ntype=rpm-md\ngpgcheck=1\ngpgkey=https://packages.microsoft.com/keys/microsoft.asc" | sudo tee /etc/yum.repos.d/vscode.repo > /dev/null
\end{lstlisting}

如果你使用较新的系统(对于 Fedora 则是 Fedora 22 或更高版本),使用这个命令来更新软件源缓存并安装 VSCode:
\begin{lstlisting}[language=bash]
    dnf check-update
    sudo dnf install code # 如果你喜欢尝鲜,把 code 换成 code-insiders
\end{lstlisting}

如果你的系统较旧,则需要使用 yum 而不是 dnf:
\begin{lstlisting}[language=bash]
    yum check-update
    sudo yum install code # 如果你喜欢尝鲜,把 code 换成 code-insiders
\end{lstlisting}

\section{在 Linux (Debian 系发行版) 下安装 Visual Studio Code}
\label{Installation_VSCode_Debian}

在 Debian 系发行版下,可以通过如下方式将 VSCode 加入软件源,从而便于安装和更新。

首先,打开终端,运行以下脚本来安装签名密钥:
\begin{lstlisting}[language=bash]
    sudo apt-get install wget gpg &&
    wget -qO- https://packages.microsoft.com/keys/microsoft.asc | gpg --dearmor > microsoft.gpg &&
    sudo install -D -o root -g root -m 644 microsoft.gpg /usr/share/keyrings/microsoft.gpg &&
    rm -f microsoft.gpg
\end{lstlisting}

随后,使用 \lstinline[language=bash]!sudo nano /etc/apt/sources.list.d/vscode.source! 创建软件源配置,将 VSCode 的软件包仓库加入系统的软件源列表。其中文件内容如下:
\begin{lstlisting}
    Types: deb
    URIs: https://packages.microsoft.com/repos/code
    Suites: stable
    Components: main
    Architectures: amd64,arm64,armhf
    Signed-By: /usr/share/keyrings/microsoft.gpg
\end{lstlisting}

最后,使用以下命令更新软件源缓存并且安装 VSCode:
\begin{lstlisting}[language=bash]
    sudo apt install apt-transport-https && sudo apt update # 更新软件源缓存
    sudo apt install code # 如果你喜欢尝鲜,把 code 换成 code-insiders
\end{lstlisting}

\section{在 MacOS 下安装 Visual Studio Code}
\label{Installation_VSCode_MacOS}