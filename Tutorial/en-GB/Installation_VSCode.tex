\chapter{Install Visual Studio Code}
\label{Installation_VSCode}

Due to limitation of FlatPak (or Snap), it is the best to install text editors for codings and IDEs like Visual Studio Code (VSCode, VSC) and JetBrains products directly on Linux system. Thus, following induction will also guide how to install Visual Studio Code to Linux directly.

\section{Install Visual Studio Code on Windows}
\label{Installation_VSCode_Windows}

There exists two different approaches to insall Visual Studio Code on Windows: via \textbf{Microsoft Store} and via \textbf{Installer}.

\subsection{Install Visual Studio Code With Microsoft Store}
\label{Installation_VSCode_Windows_Store}

If you have already logged in with your Microsoft account on your Windows machine, then install via \textbf{Microsoft Store} might be the simplist way:
\begin{enumerate}
    \item Open Microsoft Store.
    \item Search for Visual Studio Code.
    \item Choose the Visual Studio Code with blue icon, or the green one (Insider version) if you would like to enjoy new features.
    \item Click ``Install'' and wait for everything to be done!
\end{enumerate}

\subsection{Install Visual Studio Code With Installer}
\label{Installation_VSCode_Windows_Installer}

Install with installer is a traditional approach, which provides you some customisation to the Installation process.
\begin{enumerate}
    \item Visit the \href{https://code.visualstudio.com/download}{download page} of Visual Studio Code.
    \item Choose the installer for Windows. If you want more customisation, choose ``User Installer'' or ``System Installer'' below, \textbf{Remember to choose the installer that maches your CPU architecture!}
    \item Run the downloaded installer, install following its guidance.
\end{enumerate}


\section{Install Visual Studio Code in RHEL Linux Distributions}
\label{Installation_VSCode_RHEL}

In RHEL distributions, you can follow following instructions to add VSCode to your software sources, for easier installation and updates.

First, add repository and GPG key of VSCode to your system:
\begin{lstlisting}[language=bash]
    sudo rpm --import https://packages.microsoft.com/keys/microsoft.asc &&
    echo -e "[code]\nname=Visual Studio Code\nbaseurl=https://packages.microsoft.com/yumrepos/vscode\nenabled=1\nautorefresh=1\ntype=rpm-md\ngpgcheck=1\ngpgkey=https://packages.microsoft.com/keys/microsoft.asc" | sudo tee /etc/yum.repos.d/vscode.repo > /dev/null
\end{lstlisting}

If you are using newer system (for Fedora, Fedora 22 or newer), install VSCode with \texttt{dnf}:
\begin{lstlisting}[language=bash]
    dnf check-update
    sudo dnf install code # code-insiders if you want VSCode Insider
\end{lstlisting}

If you are using older system, then you need to install with \texttt{yum}:
\begin{lstlisting}[language=bash]
    yum check-update
    sudo yum install code # code-insiders if you want VSCode Insider
\end{lstlisting}

\section{Install Visual Studio Code in Debian Linux Distributions}
\label{Installation_VSCode_Debian}

In Debian distributions, you can follow following instructions to add VSCode to your software sources, for easier installation and updates.

First, run following command in terminal to install GPG key:
\begin{lstlisting}[language=bash]
    sudo apt-get install wget gpg &&
    wget -qO- https://packages.microsoft.com/keys/microsoft.asc | gpg --dearmor > microsoft.gpg &&
    sudo install -D -o root -g root -m 644 microsoft.gpg /usr/share/keyrings/microsoft.gpg &&
    rm -f microsoft.gpg
\end{lstlisting}

Then, create a new source file with \lstinline[language=bash]!sudo nano /etc/apt/sources.list.d/vscode.source!. With following content, VSCode repository can be added to system software sources:
\begin{lstlisting}
    Types: deb
    URIs: https://packages.microsoft.com/repos/code
    Suites: stable
    Components: main
    Architectures: amd64,arm64,armhf
    Signed-By: /usr/share/keyrings/microsoft.gpg
\end{lstlisting}

Finally, update software source cache, and install VSCode:
\begin{lstlisting}[language=bash]
    sudo apt install apt-transport-https && sudo apt update # Update software source cache
    sudo apt install code # code-insiders if you want VSCode Insider
\end{lstlisting}

% \section{Install Visual Studio Code on Mac}
% \label{Installation_VSCode_MacOS}