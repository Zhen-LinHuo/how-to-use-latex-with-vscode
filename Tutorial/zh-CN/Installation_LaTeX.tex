\chapter{安装 \LaTeX{}}
\label{Installation_LaTeX}

比较推荐的 \LaTeX{} 安装方式是安装 TeX Live。在其官网提供了安装的\href{https://www.tug.org/texlive/quickinstall.html}{完整指导}。

以下是一些简略的指引:
\begin{itemize}
    \item 对于 \textbf{Windows},建议直接下载 \href{https://www.tug.org/texlive/acquire-iso.html}{ISO 文件},通过挂载该镜像,执行其中的安装程序来进行安装;
    \item 对于 \textbf{RHEL 系 Linux 发行版} (如 RHEL, CentOS Stream, Rocky Linux, AlmaLinux 等),可以使用 ISO 镜像进行安装;
    \item 对于 \textbf{Debian 系 Linux 发行版} (如 Debian, Ubuntu, Kubuntu, Xubuntu, Linux Mint 等),可以直接在终端执行 \lstinline[language=bash]!sudo apt install texlive-full! 进行安装;
    \item 对于 \textbf{MacOS 和 MacOSX},建议的安装版本为 \href{https://www.tug.org/mactex/}{MacTeX},它包含全部的 TeX Live,以及一些专用于 Mac 的附加内容;
\end{itemize}

虽然 TeX Live 官方建议通过在线安装程序来安装,但是因为网络原因,除非你位于海外地区,否则使用 ISO 镜像可以有效节省安装时间。

在安装完 TeX Live 后,后续的升级就可通过 \lstinline!tlmgr update --all! 进行更新了。如果需要跨版本升级 (如从 TeX Live 2024 升级到 TeX Live 2025),则最好删除或卸载已有的更新,然后直接安装新版本。一般来说 TeX Live 几乎不需要升级。

\section{在 Windows 下安装 \LaTeX{}}
\label{Installation_LaTeX_Windows}

在下载上文提到的 ISO 镜像后,双击该 ISO 文件即可挂载。切换到挂载的镜像所在的目录,运行 \texttt{install-tl-Windows.bat},指定安装目录后采用默认设置安装即可。

依次使用以下指令查看是否有版本信息输出,从而确认安装是否成功:
\begin{lstlisting}[language=bash]
    xelatex --version
    pdflatex --version
    lualatex --version
    latexmk --version
\end{lstlisting}

\section{在 Linux (RHEL 系发行版) 下安装 \LaTeX{}}
\label{Installation_LaTeX_RHEL}

在下载上文提到的 ISO 镜像后,双击该 ISO 文件即可挂载。切换到挂载的镜像所在的目录,右键选择在该位置打开终端进入命令行。输入如下命令:
\begin{lstlisting}[language=bash]
    sudo perl ./install-tl --no-interaction
\end{lstlisting}

按照命令行提示输入密码,即可以管理员权限开始安装。

安装完成后,我们需要将 TeX Live 的路径添加到 \texttt{PATH} 以便系统和其他程序能够使用,一个可靠的方式是通过 \lstinline[language=bash]!sudo nano ~/.bashrc! 编辑当前用户的环境变量,在其下新增一行,内容为:
\begin{lstlisting}
    export PATH=<TeX Live 的安装路径>:$PATH
    export MANPATH=/usr/share/man:<TeX Live 的 man 路径>:$MANPATH
    export INFOPATH=<TeX Live 的 info 路径>:$INFOPATH
\end{lstlisting}

以在 Intel x86-64 处理器的计算机上默认安装为例,则该内容具体为:
\begin{lstlisting}
    export PATH=$PATH:/usr/local/texlive/2025/bin/x86_64-linux:$PATH
    export MANPATH=/usr/share/man:/usr/local/texlive/2025/texmf-dist/doc/man:$MANPATH
    export INFOPATH=/usr/local/texlive/2025/texmf-dist/doc/info:$INFOPATH
\end{lstlisting}

随后使用 \lstinline[language=bash]!source ~/.bashrc! 刷新环境变量,然后使用 \lstinline[language=bash]!sudo dnf install perl-core perl-Time-HiRes perl-Unicode-Normalize perl-LWP-Protocol-https! 安装必要的 perl 包,并依次使用以下指令查看是否有版本信息输出,从而确认安装是否成功:
\begin{lstlisting}[language=bash]
    xelatex --version
    pdflatex --version
    lualatex --version
    latexmk --version
\end{lstlisting}

\section{在 Linux (Debian 系发行版) 下安装 \LaTeX{}}
\label{Installation_LaTeX_Debian}

直接在终端执行 \lstinline[language=bash]!sudo apt install texlive-full! 进行安装,并依次使用以下指令查看是否有版本信息输出,从而确认安装是否成功:
\begin{lstlisting}[language=bash]
    xelatex --version
    pdflatex --version
    lualatex --version
    latexmk --version
\end{lstlisting}

\section{在 MacOS 下安装 \LaTeX{}}
\label{Installation_LaTeX_MacOS}
