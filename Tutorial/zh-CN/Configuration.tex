\chapter{配置 Visual Studio Code}
\label{Configuration}

安装完 VSCode 后,即可在已安装的程序中找到 VSCode。在配置好语言、主题、账号和同步等后,即可进入下一步:针对 \LaTeX{} 的配置。

\section{LaTeX Workshop}
\label{Configuration_LatexWorkshop}

LaTeX Workshop 是 Visual Studio Code 上颇为经典的一款扩展,提供了预览、编译选项管理、自动完成、关键词高亮等功能。这个插件也是我们选择 VSCode 而不是其他工具来编写和编译 \LaTeX{} 文档的最主要原因。

要安装 LaTeX Workshop,只需要打开左侧的扩展选项(或按快捷键 \texttt{Ctrl+Shift+X}),搜索 \texttt{LaTeX Workshop},点击安装即可。

安装完成后,点击齿轮图标进入管理菜单,选择``设置''进入其配置界面。

配置部分并不复杂。在搜索栏中已有的内容后添加 \texttt{tools},在弹出的第一个栏目下点击 ``在 \texttt{settings.json} 中编辑'',然后粘贴下述内容即可获得较为通用的配置:

\begin{lstlisting}[language=json]
    "latex-workshop.latex.autoBuild.run": "never",  // 不在保存时自动编译
    "latex-workshop.showContextMenu": true, // 启用 LaTeX 上下文菜单
    "latex-workshop.intellisense.package.enabled": true,    // 根据加载的包,自动完成命令或包
    "latex-workshop.message.error.show": false,
    "latex-workshop.message.warning.show": false,
    "latex-workshop.latex.recipe.default": "lastUsed",  // 默认将上次使用的编译方法作为默认编译动作
    "latex-workshop.view.pdf.internal.synctex.keybinding": "double-click",
    // 编译时调用的指令组
    "latex-workshop.latex.tools": [
        {
            // 采用插件预设值的 Latexmk
            "name": "latexmk",
            "command": "latexmk",
            "args": [
                "-synctex=1",
                "-interaction=nonstopmode",
                "-file-line-error",
                "-pdf",
                "-outdir=%OUTDIR%",
                "%DOCFILE%"
            ]
        },
        {
            // 全部根据 .latexmkrc 文件设置进行编译的 Latexmk
            "name": "latexmk_auto",
            "command": "latexmk",
            "args": []
        },
        {
            // 根据 .latexmkrc 文件设置进行辅助文件清理
            "name": "latexmk_auto_clean",
            "command": "latexmk",
            "args": [
                "-c"
            ]
        },
        {
            // LuaLaTeX
            "name": "lualatex",
            "command": "lualatex",
            "args": [
                "-synctex=1",
                "-interaction=nonstopmode",
                "-file-line-error",
                "%DOCFILE%"
            ]
        },
        {
            // XeLaTeX
            "name": "xelatex",
            "command": "xelatex",
            "args": [
                "-synctex=1",
                "-interaction=nonstopmode",
                "-file-line-error",
                "%DOCFILE%"
            ]
        },
        {
            // PDFLaTeX
            "name": "pdflatex",
            "command": "pdflatex",
            "args": [
                "-synctex=1",
                "-interaction=nonstopmode",
                "-file-line-error",
                "%DOCFILE%"
            ]
        },
        {
            // Biber
            "name": "biber", 
            "command": "biber",
            "args": [
                "%DOCFILE%"
            ]
        },
        {
            // BibTeX
            "name": "bibtex",
            "command": "bibtex",
            "args": [
                "%DOCFILE%"
            ]
        }
    ],
    "latex-workshop.latex.recipes": [
        {
            // 完全根据 .latexmkrc 文件进行编译
            "name": "Latexmk (auto)",
            "tools": [
                "latexmk_auto"
            ]
        },
        {
            // 与上一个对应,用于清理输出目录的辅助文件
            "name": "Latexmk Clean (auto)",
            "tools": [
                "latexmk_auto_clean"
            ]
        },
        // 各个编辑器和 Biber 组合,用于含目录和参考文献的场景
        {
            "name": "lualatex -> biber -> lualatex*2",
            "tools": [
                "lualatex",
                "biber",
                "lualatex",
                "lualatex"
            ]
        },
        {
            "name": "xelatex -> biber -> xelatex*2",
            "tools": [
                "xelatex",
                "biber",
                "xelatex",
                "xelatex"
            ]
        },
        {
            "name": "pdflatex -> biber -> pdflatex*2",
            "tools": [
                "pdflatex",
                "biber",
                "pdflatex",
                "pdflatex"
            ]
        },
        // 各个编译器和 BibTeX 的组合,用于含目录和参考文献的场景
        {
            "name": "lualatex -> bibtex -> lualatex*2",
            "tools": [
                "lualatex",
                "bibtex",
                "lualatex",
                "lualatex"
            ]
        },
        {
            "name": "xelatex -> bibtex -> xelatex*2",
            "tools": [
                "xelatex",
                "bibtex",
                "xelatex",
                "xelatex"
            ]
        },
        {
            "name": "pdflatex -> bibtex -> pdflatex*2",
            "tools": [
                "pdflatex",
                "bibtex",
                "pdflatex",
                "pdflatex"
            ]
        },
        // 单一编译器命令
        {
            "name": "LuaLaTeX",
            "tools": [
                "lualatex"
            ]
        },
        {
            "name": "XeLaTeX",
            "tools": [
                "xelatex"
            ]
        },
        {
            "name": "PDFLaTeX",
            "tools": [
                "pdflatex"
            ]
        },
        {
            "name": "Latexmk",
            "tools": [
                "latexmk"
            ]
        },
        {
            "name": "Biber",
            "tools": [
                "biber"
            ]
        },
        {
            "name": "BibTeX",
            "tools": [
                "bibtex"
            ]
        }
    ],
\end{lstlisting}

如此,在编写好你的 \LaTeX{} 文档后,只需转到 LaTeX Workshop 的侧边栏页面,执行一次编译指令,后续点击 \texttt{.tex} 文件编辑窗口右上角的构建运行按钮(\texttt{Ctrl+Alt+B}),即可一键编译为 PDF 了。

\section{进阶技巧:Latexmk}
\label{Configuration_Latexmk}

就如同前文中设置 Recipe 时所展现的一样,如果文章有目录,你需要连续用同一个编译器编译两次才能获得正确的 PDF;而如果文章有参考文献,还需要先编译一次,再经过 Bibier 或 BibTeX,再由同一个编译器编译一次(如果文章含目录,则需要再编译两次)才可以。有时,为应对复杂的功能需求,甚至还需要在编译时运行一些外部程序。随着需求越来越复杂,显然只是配置 Recipe 会让选项越来越多且不一定实用。

Latexmk 是一个 Perl 脚本,它可以为你处理所有这些事情。\href{https://mgeier.github.io/latexmk.html}{这里}是我发现的一个介绍页面,介绍了如何直接使用 latexmk。

Latexmk 还可以配合对应的 \texttt{.latexmkrc} 文件来管理运行选项。这也是之前的配置文件中我们添加了 \texttt{Latexmk (auto)} 和 \texttt{Latexmk Clean (auto)} 这两项的原因。前者将另 latexmk 完全根据你的项目目录的(如果没有,则会参考当前用户的或者系统的)\texttt{.latexmkrc} 文件中的设置来编译生成文档;后者则用于对应的执行清理辅助文件的工作。

以下是关于 \texttt{.latexmkrc} 中常用的几种设置的介绍。可以参考本项目中采用的 \texttt{.latexmkrc} 来进一步了解各个设置的具体用法。

\paragraph{设置输出和编译器}
\label{Configuration_Latexmk_Compiler}

使用 \lstinline[language=perl]!$pdf_mode! 设置是否输出 PDF 文件:
\begin{itemize}
    \item 0:   不生成 PDF
    \item 1:   使用 PDFLaTeX 编译
    \item 2:   使用 PS2PDF 编译
    \item 3:   使用 DVIPDF 编译
    \item 4:   使用 LuaLaTeX 编译
    \item 5:   使用 XeLaTeX 编译
\end{itemize}

其中,PDFLaTeX 生成很快, XeLaTeX 和 LuaLaTeX 对 UTF-8 支持更好,都是比较常用的选项。

对于参考文献工具,使用 \lstinline[language=perl]!$bibtex_use! 选项来控制。
\begin{itemize}
    \item 0:   不使用 BibTeX 和 Bibier,也不会清除 \texttt{.bbl} 文件
    \item 1:   仅当 \texttt{.bib} 文件存在时使用 BibTeX 或 Bibier,不会清除 \texttt{.bbl} 文件
    \item 1.5: 仅当 \texttt{.bib} 文件存在时使用 BibTeX 或 Bibier,仅当 \texttt{.bib} 文件存在时清除 \texttt{.bbl} 文件
    \item 2:   仅当 \texttt{.bib} 文件存在时使用 BibTeX 或 Bibier,无论 \texttt{.bib} 文件是否存在,都会清除 \texttt{.bbl} 文件
\end{itemize}

此外,对于 Bibier,一般使用如下的参数配置:\lstinline[language=perl]!biber = "biber %O %S";!

\paragraph{设置主文件}
\label{Configuration_Latexmk_MainFile}

可以通过 \lstinline[language=perl]!@default_files! 设置需要编译的主文件.需要忽略的文件则可通过 \lstinline[language=perl]!@default_excluded_files! 设置。一个可供参考的范例如下:
\begin{lstlisting}[language=perl]
    @default_files = ("build-en-GB.tex", "build-zh-CN.tex");
    @default_excluded_files = ();
\end{lstlisting}

\paragraph{输出路径和清理}
\label{Configuration_Latexmk_OutDirAndClean}

可以通过 \lstinline[language=perl]!$out_dir! 设置输出文件所在的目录。编译中产生的辅助文件也会放在那里。

通过 \lstinline[language=perl]!$out_dir! 设置需要清理的中间文件。选择性的保留中间文件对需要发表到期刊的论文可能很有帮助。